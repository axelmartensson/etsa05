
%% report_template.tex
%% V1.0
%% 2012-03-16
%% by Jesper Pedersen Notander
%% See:
%% http://www.cs.lth.se/jesper_pedersen_notander
%% for current contact information.
%% V1.1
%% 2014-02-20
%% by Per Runeson
%% See:
%% http://www.cs.lth.se/per_runeson
%% for current contact information.
%%
%% This is a template file contaning instructions and a skeleton outline 
%% for the final report in the course ETSA05: Software Engineering 
%% Process - Soft Issues, given by the Department of Computer Science at 
%% Lund University, Sweden.
%% 
%% This template requires IEEEtran.cls, written by Michael Shell, version 
%% 1.7 or later.
%%
%% Support sites:
%% http://www.cs.lth.se/etsa05/
%% http://www.ieee.org/

%%*************************************************************************
%% Legal Notice:
%% This code is offered as-is without any warranty either expressed or
%% implied; without even the implied warranty of MERCHANTABILITY or
%% FITNESS FOR A PARTICULAR PURPOSE! 
%%
%% User assumes all risk.
%%
%% In no event shall Lund University or any contributor to this code be 
%% liable for any damages or losses, including, but not limited to, 
%% incidental, consequential, or any other damages, resulting from the use 
%% or misuse of any information contained here.
%%
%% All comments are the opinions of their respective authors and are not
%% necessarily endorsed by Lund University.
%%
%% This work is distributed under the LaTeX Project Public License (LPPL)
%% ( http://www.latex-project.org/ ) version 1.3, and may be freely used,
%% distributed and modified. A copy of the LPPL, version 1.3, is included
%% in the base LaTeX documentation of all distributions of LaTeX released
%% 2003/12/01 or later.
%%
%% Retain all contribution notices and credits.
%% ** Modified files should be clearly indicated as such, including  **
%% ** renaming them and changing author support contact information. **
%%
%% File list of work: report_template.tex
%%*************************************************************************


\documentclass[conference]{IEEEtran}
% If IEEEtran.cls has not been installed into the LaTeX system files,
% manually specify the path to it like:
% \documentclass[conference]{../sty/IEEEtran}
\usepackage{hyperref}
\begin{document}

\title{Outline}


% author names and affiliations
% use a multiple column layout for up to three different
% affiliations
\author{\IEEEauthorblockN{Kasper Bratz}
\IEEEauthorblockA{Department of Computer Science\\
Faculty of engineering, LTH
Lund, Sweden\\
dat12kbr@student.lu.se}\and

\IEEEauthorblockN{Axel Martensson}
\IEEEauthorblockA{Department of Computer Science\\
Faculty of engineering, LTH
Lund, Sweden\\
dat12ama@student.lu.se}\and

\IEEEauthorblockN{Christine Boghammar}
\IEEEauthorblockA{Department of Computer Science\\
Faculty of engineering, LTH
Lund, Sweden\\
dat12cbo@student.lu.se}\and

\IEEEauthorblockN{Johan Nilsson}
\IEEEauthorblockA{Department of Computer Science\\
Faculty of engineering, LTH
Lund, Sweden\\
dat12jn1@student.lu.se}}


\maketitle


\begin{abstract}
Facebook has over the last decade segmented itself as the leading social
network on the internet. This paper analyzes the accessibility, ethical, legal
and business aspects of facebook. 

Method for the study, results and conclusions must be written. 

\end{abstract}

\section{Introduction}
Facebook is a social networking platform created by Mark Zuckerberg in 2004.
It was originally designed as a way to communicate and share information for
students attending Harvard but has today expanded to accept people from all
over the world. A more detailed description of it is provided in the following
section. In this paper facebook is viewed as an ICT system. Using the ISO 9126
standard the main quality characteristics of facebook are identified. This
prompts a discussion of the sub-characteristics where Facebook is found
lacking.  One such characteristic of special interest is the accessibility for
disabled people, especially the blind. The problems arising from the use of an
assistive technology called a screen reader are analyzed and a possible
solution for making Facebook more screen reader accessible is highlighted.
Facebook thrives on user generated content, however the handling of this
content is non-transparent. The ethical concerns, notably those related to the
privacy of it's users and to the user generated content are identified and
discussed. In contrast to this; When it comes to protecting the privacy of the
company, they devote a lot of resources to searching the web for violations of
their intellectual property.  Software code is hard to protect under current
intellectual property laws, Facebook's strategy for doing so is examplified
through two law suits and the ensuing discussion aims to provide a deeper
understanding of how large software companies protect themselves against
lawsuits. Not getting sued is a good way to secure the future of the company,
but to be a truly successful company one also has to make a profit. The
business model of Facebook is discussed and the reason that it is so radically
dominant in the area of social network platforms is sought out. The conclusions
from this discussion is used as a backdrop for analyzing how their progress
derives from their business model. Based on a review of the relevant
literature, this paper analyzes the accessibility, ethical, legal and business
aspects of facebook. 

\section{Description of the System}
\label{system}

Facebook is a social networking platform created by Mark Zuckerberg in 2004.
Facebook originally only accepted students attending Harvard but has today
expanded, allowing user from all over the world. Today facebook has
approximately 1.2 billion monthly users reference according to compete.com and
has now taken the spot as the number one sociall network of the world.

Facebook is a way for people to share thoughts, pictures and video to friends
and family by posting them on your personal profile. Your profile also contains
a so called “News feed” where updates posted by people on your friends list
will be displayed. Users are also able to create fan pages, these are open
pages and liking one of these pages makes updates from said page show up on
your news feed without. This gives people with the same interests a way to
share ideas and opinions about a subject.


Facebook also features several other well used features like:
“The Facebook chat”: where about  3 million messages are sent every minute.
An event calendar: which is a digital calendar where you are able to create and
invite people to events. You can also create public events that are open for
everyone to join.

//Kvar att skriva om: Appar, grupper(?) lite mer almänt om vad facebook kan
användas till. Exempelvis för att organisera händelser osv.  


\section{Quality Characteristics}
\label{quality}
From seminar 1 we considered the main characteristics for facebook to be
usability, functionality and portability and as we’re doing more research this
seems to hold true. We will give further understanding as to why this is the
case, by focusing on what users prioritize when using facebook. At this point
facebook has grown so large, that people will use the product regardless if it
has a few bugs. We will also give examples on how facebook is faulty in other
characteristics, mainly the reliability, security and understandability(Thomas
Paul et.al).


\section{Availability for Disabled}
\label{availability}
%:set fo=tqa

% Why are we focusing on the blind?
% What types of accessibility products exist for the blind?
% Which problems are encountered by people accessing facebook with a screen reader today?
% What can be done to improve accessibility of facebook when using a screen reader?

This section will focus on the availability of facebook for blind computer
users. The decision to do this stems from the result of a study conducted by
Petrie et al.\cite{petrie2004}. It was performed using a user panel that
evaluated the accessibility of 100 web pages.  The user panel was comprised of
people with various disabilities affecting their access to the web, such as
blindness and dyslexia. User panel members were told to evaluate 10 websites
and perform two representative tasks on each of them. The blind panel group had
the lowest success rate of all the disability groups represented in the study.
Their success rate was 53\% compared to an average success rate of 82\%.  This
result motivates further discussion of the tools available to a blind computer
user when browsing on the internet.

Blind computer users use a piece of software called a screen reader.
JAWS\cite{jaws} is one example for Windows users. It has a navigation mode for
browsing the web and an editing mode for filling out forms. The purpose of the
screen reader is to read the text currently being displayed on the screen and
then convey this information via speech synthesis to the computer user. Screen
readers read a page of text by simply starting to read the page sequentially
from left to right, top to bottom. The user then has to use this information to
build a mental model of what is presently on the screen and how it is laid out.
The blind user is of course completely dependent on the feedback provided by
their screen reader for interacting with software on the computer. 

Screen readers have not been able to keep up with the advance of web 2.0
technology and sites like Facebook that makes heavy use of such technology. The
front-end provided by these websites is to a large extent powered by
JavaScript.  Common obstacles for screen readers that arise from the use of
Javacript are custom controls\cite{ariaprimer} that are built with DOM-elements
that contain no semantic information, such as DIV-tags. These are then
decorated by javascript to provide functionality.  However, without semantic
information, how is the screen reader supposed to interpret a given DIV tag?  

The W3C’s WAI-ARIA standard\cite{ariaspec} could be used to solve this
problem\cite{ariaprimer}.  It defines an interface for providing semantic
information through the addition of more classes to the DOM.  The standard is
very new and it is referenced at least once in all the papers cited in this
section. It is important to note that none of them provide any data on its use.
Furthermore, the standard  is not implemented in JAWS\cite{hailpern2009}.  Its
most relevant contribution for the sake of this discussion is that it
introduces the notions of role and state for elements of the DOM.  Consider the
case where an element looks like a link and/or behaves like a link, it should
then be given the WAI-ARIA role `link' to enable screen readers to interpret it
as such.  Hailpern et al\cite{hailpern2009}.  suggest that to make custom control
elements accessible to screen readers, they should have a keyboard listener, be
added to the tabindex and have an appropriate WAI-ARIA role.

Another feature of JavaScript is that it  enables dynamic updates of web pages.
This means that it increases the performance of web applications by not making
the browser reload the whole web page when something changes but instead
limiting the reload to a certain region of the page. Screen readers have not
yet caught up with this shift from static web pages with only text links and a
few images here and there to dynamic web content. They do not detect changes
that happen dynamically to the content (i.e after a page has
loaded)\cite{hailpern2009}.  When an event occurs on Facebook, e.g a new friend
request is recieved, the user is notified by a little red icon that appears on
the corresponding button in the topmost navigation bar.  This change is not
detected by the screen reader\cite{buzzi2010} because it is made dynamically by
a script.  The only time that an existing screen reader detects a change is
when doing a page refresh. A side effect of doing a refresh is that the screen
reader loses its current place on the page and then the user has to re-read the
whole page to get to where the change occurred. This can be a great source of
frustration for the user.\cite{petrie2004} 

Support for dynamic updates can be achieved by a combination of two
things\cite{hailpern2009}: First, developers must provide audio feedback on
event, e.g a voice saying ''You've got a new friend request'' when someone
sends you a friend request. Secondly, WAI-ARIA has a concept called ``Live
region''\cite{ariaspec} that enables web developers to mark certain elements
(e.g a table) as dynamically updating. This is meant to enable the screen
reader to keep an eye out for changes happening dynamically in this region of
the page.

To provide a convenient way to access Facebook and other web 2.0 sites,
Hailpern et al\cite{hailpern2009}.  propose an alternative screen reader use
model with a reading mode and a control mode instead of the modes mentioned
above. The workflow of this use model is best illustrated by an example: When
receiving a new private message on Facebook use control mode to navigate to the
inbox and open the message, then use reading mode to read the message.  We
argue that the main advantage of this use model is the increased cohesion
between common application design patterns such as Model-View-Controller and
the use model of the screen reader, thereby making the screen reader more
intuitive to use on a web 2.0 site. It is important to keep in mind that
Facebook is a large website and it has different parts of it that are more or
less intuitive and accessible.

A more nuanced impression of facebook's accessibility is given in Meiselwitz'
and Lazar's \cite{meiselwitz2009} study of the accessibility of Facebook's
sign-up process. They conclude that the sign-up process is ``moderately
inaccessible'' and that it only violates a single paragraph in Section 508 of
the US Rehabilitation Act, \emph{which} paragraph is not mentioned. This study
was conducted using only automated testing tools which is why we regard these
results with a certain scepticism because they fail to encompass the user's
experience of navigating the site, unlike \cite{buzzi2010}.

We conclude from our review of the literature that Facebook still has a long
way to go before it can be comfortably deemed accessible for blind computer
users.


% % Problems Dynamic content updates: Notifying the user of a change when
% dynamically updating a web page. e.g a new friend request. Now: screen
% readers can only detect that something has changed on page refresh.
% Furthermore when user is not provided feedback by screen reader when
% performing an action such as clicking a link.  This results in the user
% having to painstakingly re-read the whole page to discover the change.
% 
% Custom controls: decorated links, using divs for various stuff, javascript.
% 
% Managing focus: what is important on this page?
% 
% % Solutions
% 
% Dynamic content updates: Live Regions. Developers provide audio feedback on
% event, e.g a voice saying ''You've got a new friend request'' when someone
% sends you a friend request.  
% 
% Custom controls: should have keyboard listener and be added to the tabindex. If
% it looks like a link and behaves like a link, it should have the ARIA role
% `link'.\cite{hailpern2009}
% 
% Alternative User workflow model.  use a reading mode for traditional web
% browsing and a control mode for interacting with the application instead of
% navigation mode and forms mode\cite{hailpern2009}. e.g upon receiving a new
% private message on Facebook use control mode to locate and open the message,
% then use reading mode to read the message.

\section{Ethical Aspects}
\label{ethics}
As previously mentioned Facebook has approximately  1.2 billion monthly users
which means Facebook has access to names, email-address, date of birth and your
gender since this is mandatory information for signing up a Facebook account.
Most people also submit their phone numbers and addresses. So what is Facebook
actually allowed to do with this information?

What can you post on Facebook?
In Italy, pro maffia groups were discovered which caused the Italian senate to
pass a bill that says that Italian ISP’s have to block access to websites
refusing to remove content that might be used to plan criminal activities. Both
Facebook and Google have criticized this decision saying this would be against
the freedoms of speech if the person in question actually wasn't planning
anything illegal.  Having this law passed would as Facebook and Google pointed
out enable the government to censor whoever they would want by claiming that
the person was planning something illegal. 

More to discuss:
What can fb do with your information, who can see what? privacy policy, crime,
online violence.

\section{Legal Aspects}
\label{legal}
In this section we plan on writing about the intellectual properties of
facebook. Being the leading company in its field, facebook has a large focus on
searching the web for violations of their intellectual property. This concern
originates from the fact that software code is a very difficult thing to
protect. According to  Xuan-Thao Nguyen et.\ al.\ this leads to facebook
protecting its software code through trade dress lawsuits and so on. We will
continue by providing examples of the different lawsuits that facebook has gone
through over the last couple of years, and lastly an overview of how facebook
protects itself from the information that their users post will be given.

\section{Business Aspects}
\label{financial}

introduction:
This submission describes the success of Facebook from a business point of view
considering the fact the company being the leader in social network platforms.
Looking at Facebook today it is often said that it has the most effective and
mature business model compared to other highly developed companies. The
question “what makes Facebook so radically dominant in the area of social
network platforms and how their progress derives from their business model?
will be answered in this section. To give a broader understanding, an
explanation will be given of the concept and meaning of having a model
containing business aspects such as financial factors, targeted customers and
company values.

1. The concept of a business model
A business model purpose is to give an outline about the product/service that's
being offered. The point is to think through, validate and analyse all
different significant elements to make the business logic as presentable as
possible to possible stakeholders. It should conclude elements as target
customers and value propositions. It is important to put a lot of effort into
cost structure detailing revenue streams and profit margins since these are
generally the key for possibilities for a company of moving forward and to
develop even further. 
To pinpoint some key questions that is quite straight forward in a business model:
Who are the target customers?
How will it reach, acquire, and keep customers?
How will it be defined and differentiate your offering?
How will it generate revenue?

Considering a business model in an entrepreneurial setting it is used as
described earlier but it can also be used as something to lean back on to when
managers inside companies are exploring possibilities of taking the current
core of the business to further development and broadening the concept. Looking
at the meaning and benefits of an organized and well defined business plan as
it has been described it is now accounted for that it can be used as a
mechanism for developers making extensive growth in firms and successful
businesses.  

2. Target customers and values of Facebook
Facebook has primary two target customers, private users and business users.
Even though the factor of business users are growing each day the rate of
private users are growing even more. The elusive social network platform is
providing a broad range of space and an environment for the sector of private
users since it was originally originated for them as it was at first built for
students at harvard as a social community. The founder Mark Zuckerberg felt a
platform for students to communicate and to reach out to new friends was a
concept that was needed in that surrounding he was in. What he did not know was
that it was something needed in the whole world. As he helped connecting the
world in all kinds of corners it became an attractive forum for companies to
reach out to potential customers. Since users are already given out their
information and interests Facebook has all the tools needed to know where
comercial should be placed to reach the meant targeted customers. Other
companies who do not have this privilege has to use other resources to extract
information about their targeted customers and therefore have to put more work
and money into it. Obviously this is profitable for both parts being Facebook
and advertisers. While Facebook is a free service one problem Facebook had to
work around was how to get revenue from advertisers but still making Facebook
an exciting experience for their users since having ads popping up while you
are connected is not a desirable experience. On one hand Facebook users want to
be able to access information about products by advertisers but on the other
hand do not want to be felt like it is being forced upon them. 

The concept of Facebook is as mentioned a platform to share, connect and reach
out to friends and the mainly mission is to provide information to the users so
the information wished to be shared actually get to be seen by friends and
intended regions. As Facebook´s one value is to make a bridge between other
channels of communication and information into Facebook so users do not have to
leave their page to get the desired information from outside sites. Integrating
with other platforms such as news sites, providing users with news while
connecting with their daily friends and pages has led the users tending to use
Facebook as an substitute to other news-portals. Another channel of information
Facebook now has integrated with is Google which now has added Facebook pages
updates to their real-time search. In result making it easy to access relevant
information on Facebook from searches made on google. The consequences of this
concept broadening is the factor of wide range of context is put into the
virtual space concentrating it to Facebook creating a lock-in effect on the
users. 

    
Additionally to bridging channels of information, primarily private users,
Facebook has also further developed the platform in offering virtual games to
the users. This way users tend to spend more time on the site and hopefully
reach out and catch the attention of a new group of potential users.

3. Facebook being a service platform
Facebook´s rapidly growth had led it to become the biggest social network
platform with a number of 1.28 billions user. 1.23 billion is using Facebook
each month and 757 million of them is connected every day. Thinking of these
statistics gives a perspective of how big the service platform is. Being a
service platform you would think that the employes working for Facebook are the
ones supplying with the service but since Facebook is a platform providing the
functionality of presented the service the user needs. Considering all kinds of
input has a value in it, the amount of input and information that approximately
1,2 billion users feed to Facebook voluntarily is great resource and value.
Building a network where the users do all the work for them. Due of the great
number of users there is no doubt that it is a popular platform for companies
to advertise on. Companies and firms can now set up a page for the intention of
commercial means and to be able to an up-to-date feed of information about
what's going on with the company . This way the information is out there but
the user can choose to look it up. The great invention of Facebook´s “Like
buttom” gives the users the possibility to like a page and get notifications
pushed to them as the pages writes updates. On the other hand since Facebook
has the information of their users fed to them they can easily promote
different pages to users likely to be interested in what the page is about.
Statistics show that 64% of US top 50 law firms have a Facebook page and 25
million number of small business have a page on Facebook. A total number of 7
million of apps and websites are integrated with Facebook which indicates the
vast range Facebook comes through on the internet. As this goes on Facebook
keeps delivering functionality to their broad range of users and is frequently
launching new features to the platform.

4. The financial aspects
Up until two years ago Facebook did not publicly disclose its financial
results but on Friday the 18th of May 2012 was the initial public offering(IPO)
held. Although the IPO almost closed at opening price at 100 billion dollar it
still was one the biggest peak of capitalization in technology and internet
history. The platform being Facebook reached a revenue in 2013 of an amount
consisting 7.87 billion dollars. Its revenue increased by 55% from the year
2012 to 2013. 
Starting with Facebooks cost structure it consists mainly of four different
cost areas; data center, marketing, administrative and research costs. As
mentioned before Facebook has the privilege of the users doing most of the work
for them, although a company so big should entail a grand number of employees.
As of March 2014 Facebook had 6,818 employees and that is simply implying how
agile the company infrastructure must be since so little resources is needed. 
Moving on to the revenue streams as known it is free to join the social
network, you could say that it is partly a stream of revenue reasoning the fact
without users wanting to connect to Facebook there is no way to get earning out
of it. The concept of Facebook is to connect to each other and not exactly to
be connected to outside internet sites if it is not intended of the user.
Considering google as an example on the counterpart. When a user enters google
search engine they already have a purchase intent and is wanting to look up
products with ads and commercial. Facebook users on the other hand is usually
not in a intent of purchasing. Ads reduce the users experience and satisfaction
by trying to turn the users from a non purchase intent to wanting to click on
the ads and therefore making a profit for Facebook and the same time satisfying
the advertisers. How can Facebook not displaying ads and making a poor
experience for the users and still make profit by not trying to force users to
convert from non purchase to an purchase intent? One way Facebook, being this
huge service platform as it is, came up with the concept of offering a game
platform for developers connecting Facebook users with developers produced apps
and games. The applications such as games have to be free to the users the
in-app-purchases however do not have to be. Apps involving games is also very
popular to play at Facebook's mobile app. The statistics state that 59% of
Facebook revenue comes from mobile platforms and the number of in-game
purchases made in 2012 was 27 million. The Amount of hours per month users
spend on playing Facebook games is 927 million hours or approximately 106.000
years/month.
This way Facebook and developers get a win-win situation such as revenue and
branding of developers applications and generates an income to Facebook. Of
course Facebook realised that the income that would come from ads could be of
great value since the ads on their platform would be seen by a great amount of
the 1,2 billion Facebook users. The share of mobile internet revenue worldwide
is 17,5% in 2013. As mentioned in previous section Facebook has the tools to
know what kind of user to target with specific ads and therefor not hurting the
experience that much for the users. In some cases to point ads in the right
direction even might increase the satisfaction of the user surfing on Facebook
especially when trying to merge more channel of communication and service into
one. 

Summary of my part on business aspects..
Yo sum up this section:
Often mentioned as the greatest business plan ever seen becouse..
- the users do the work.. providing with information about interests and
connections.
- agile, inte många utgifter och få anställda.. 
- utnyttjar nätverket till att få andra att dra till sig användare.. 
börjar likna google på det plan att de har information om företag och produkter
som kan hittas på facebook. 
Channels:
Websites, mobile apps
Fb apps and pages
Developers tools and API
De låter andra hemsidor integrera med deras funktioner så som like knappar.


\section{Summary} 
- 

\section{Contribution Statement} 

Johan Nilsson - Quality
Characteristics, Legal aspects.

Christine Boghammar – Business aspects.

Axel Mårtensson - Availability for Disabled

Kasper Bratz - Ethical aspects, Description of the System

Abstract
Introduction
Summary
% references section
\begin{thebibliography}{1}

\bibitem{jung2004}
H-W. Jung, S-G Kim, and C-S Chung, ''Measuring Software Product Quality: A
Survey of ISO/IEC 9126'' IEEE Software, 21(5), pp. 88--92, 2004.

\bibitem{berenbach2004}
B. Berenbach, and M. Broy, ''Professional and ethical dilemmas in software
engineering'' Computer, 42(1), pp. 74--88, 2004

\bibitem{nguyen2012}
Xuan-Thao Nguyen, and Jeffrey A. Maine, “Taxing facebook Code: Debugging the
code and software”, Buffalo Law Review, January 2012(Peer Reviewed)

\bibitem{lilley2012}
Stephen Lilley, Frances S. Grodzinsky, Andra Gumbus, (2012) "Revealing the
commercialized and compliant Facebook user", Journal of Information,
Communication and Ethics in Society, Vol. 10 Iss: 2, pp.82 – 92 (Peer Reviewed)

\bibitem{cabrera2011}
Jan André Blackburn-Cabrera, “The war on Facebook: Privacy on social networks”,
International Journal of Liability and Scientific Enquiry(IJLSE), Vol. 4, No.
4, 2011 (Peer Reviewed)

\bibitem{hensley2012}
Jason Hensley, 2012, “Facebook success shows why Intellectual Property laws are
stifling”,6/2 - 2012,
\href{http://www.policymic.com/articles/3939/facebook-success-shows-why-intellectual-property-laws-are-stifling}{http://www.policymic.com/articles/3939/facebook-success-shows-why-intellectual-property-laws-are-stifling}

\bibitem{brodkin2012}
Jon Brodkin,  2012, “Yahoo IP lawsuit: We patented “Facebook’s entire social
network model”, 13/3 - 2012,
\href{http://arstechnica.com/tech-policy/2012/03/yahoo-patent-lawsuit-we-invented-facebooks-entire-social-network-model/}{http://arstechnica.com/tech-policy/2012/03/yahoo-patent-lawsuit-we-invented-facebooks-entire-social-network-model/}

\bibitem{delamerced2012}
Michael J.De La Merced, 2012, “Yahoo and Facebook settle patent lawsuits”, 6/6
– 2012,
\href{http://dealbook.nytimes.com/2012/07/06/yahoo-and-facebook-said-to-settle-patent-lawsuits/?\_php=true\&\_type=blogs\&\_php=true\&\_type=blogs\&\_r=1}{http://dealbook.nytimes.com/2012/07/06/yahoo-and-facebook-said-to-settle-patent-lawsuits/?\_php=true\&\_type=blogs\&\_php=true\&\_type=blogs\&\_r=1}

\bibitem{epo2013}
European Patent Office, 2013, “Patents for software? European law and
practice”, 26/11 – 2013, \href{http://www.epo.org/news-issues/issues/software.html}{http://www.epo.org/news-issues/issues/software.html}

\bibitem{facebookrights2013}
Facebook, - , “Statement of Rights and Responsibilities”, 15/11 - 2013
https://www.facebook.com/legal/terms

\bibitem{yulianti2013}
Ida Yulianti, and Wei Tung, 2013, “The Relationship Among Brand Experience,
Brand Image and

\bibitem{indonesia}
Customer Satisfaction of Facebook Users in Indonesia“, 1818-4952, Department of
Management, Faculty of Economic and Business, and University of Brawijaya 1

\bibitem{chiayi}
Graduate Institute of Marketing and Logistics, National Chiayi University,
\href{http://www.idosi.org/wasj/wasj28(efmo)13/15.pdf}{http://www.idosi.org/wasj/wasj28(efmo)13/15.pdf }

\bibitem{paul}
Thomas Paul, Daniel Puscher, and Thorsten Strufe, “Improving the Usability of
Privacy settings in Facebook”, \href{http://arxiv.org/abs/1109.6046}{http://arxiv.org/abs/1109.6046 }

\bibitem{kumparak2013}
Greg Kumparak, 2013, “Security Researcher Hacks Mark Zuckerberg’s Wall To Prove
His Exploit Works”, 18/8 - 2013,
\href{http://techcrunch.com/2013/08/18/security-researcher-hacks-mark-zuckerbergs-wall-to-prove-his-exploit-works/}{http://techcrunch.com/2013/08/18/security-researcher-hacks-mark-zuckerbergs-wall-to-prove-his-exploit-works/}

\bibitem{priestley2011}
*Is this a valid source?*Evan Priestley, 2011, “Is it true that facebook has no
testers”, 20/12 – 2011,
\href{http://www.quora.com/Is-it-true-that-Facebook-has-no-testers}{http://www.quora.com/Is-it-true-that-Facebook-has-no-testers}

\bibitem{priestley2011b}
*Is this a valid source?*Evan Priestley, 2011, “Why does Facebook not need to
produce particulary high-quality code?”, 20/12 – 2011,
\href{http://www.quora.com/Facebook-Engineering/Why-does-Facebook-not-need-to-produce-particularly-high-quality-software}{http://www.quora.com/Facebook-Engineering/Why-does-Facebook-not-need-to-produce-particularly-high-quality-software}

\bibitem{innovation}
(?)Buisness Model Innovation Matters,
\href{http://bmimatters.com/2012/01/08/understanding-business-model-fundamentals/}{http://bmimatters.com/2012/01/08/understanding-business-model-fundamentals/}

\bibitem{weiner2010}
Nico Weiner, 2010 , “Social networks evolving into service platforms - the

\bibitem{leipzig2010}
Facebook-case from a business model viewpoint”, 40th Jahrestagung der
Gesellschaft fur Informatik e.V. (GI): Service Science - Neue Perspektiven fur
die Informatik, INFORMATIK 2010, Leipzig, Germany, September 27, 2010 - October
1, 2010, \href{http://subs.emis.de/LNI/Proceedings/Proceedings175/527.pdf}{http://subs.emis.de/LNI/Proceedings/Proceedings175/527.pdf}

\bibitem{weiner2011}
Nico Weiner, 2011, “A Business Model Framework for the Design and Evaluation of
Business Models in 

\bibitem{srii}
the Internet of Services ”, Annual SRII Global Conference, Stuttgart, Germany,
-,
\href{http://ieeexplore.ieee.org.ludwig.lub.lu.se/stamp/stamp.jsp?tp=\&arnumber=5958069\&tag=1}{http://ieeexplore.ieee.org.ludwig.lub.lu.se/stamp/stamp.jsp?tp=\&arnumber=5958069\&tag=1}

\bibitem{hof2011}
Robert D. Hof, 2011, “You are the ad”, 1099-274X, -,
\href{http://eds.a.ebscohost.com.ludwig.lub.lu.se/eds/detail?sid=61403bdf-588f-4bf7-be69-d67225c27c5b\%40sessionmgr4001\&vid=3\&hid=4208\&bdata=JnNpdGU9ZWRzLWxpdmUmc2NvcGU9c2l0ZQ\%3d\%3d\#db=bth\&AN=61481061}{http://eds.a.ebscohost.com.ludwig.lub.lu.se/eds/detail?sid=61403bdf-588f-4bf7-be69-d67225c27c5b\%40sessionmgr4001\&vid=3\&hid=4208\&bdata=JnNpdGU9ZWRzLWxpdmUmc2NvcGU9c2l0ZQ\%3d\%3d\#db=bth\&AN=61481061}

\bibitem{ortutay2014}
Barbara Ortutay, 2014, “Facebook: Born in a dorm room, the site that had no
business plan turns 10 today and is worth \$150 billion”, 4/2 – 2014
,\href{http://www.vancouversun.com/touch/story.html?id=9468642}{http://www.vancouversun.com/touch/story.html?id=9468642}

\bibitem{haigh2013}
Michel M. Haigh, Pamela Brubaker, Erin Whiteside, (2013) "Facebook: examining
the information presented and its impact on stakeholders", Corporate
Communications: An International Journal, Vol. 18 Iss: 1, pp.52 – 69

\bibitem{krombholz2012}
Katharina Krombholz et. Al, 2012, “Fake identities in social media: A case
study on the sustainability of the Facebook business model”, 2093-0739, SBA,
Research, Vienna, Austria (et.\ al)

\bibitem{mourdoukoutas}
Panos Mourdoukoutas, 2014, “What's wrong with Facebook's Business model and
Innovation strategy?”, 23/2 - 2014,
\href{http://www.forbes.com/sites/panosmourdoukoutas/2014/02/23/whats-wrong-with-facebooks-business-model-and-innovation-strategy/}{http://www.forbes.com/sites/panosmourdoukoutas/2014/02/23/whats-wrong-with-facebooks-business-model-and-innovation-strategy/}

\bibitem{dixon2012}
Chris Dixon, 2012, “Facebook's business model”, 15/5 – 2012,
\href{http://www.cdixon.org/2012/05/15/facebooks-business-model/}{http://www.cdixon.org/2012/05/15/facebooks-business-model/}

\bibitem{owyang2012}
Jeremiah Owyang, 2012, “Facebook: A brilliant Business model”, 19/5 - 2012,
\href{http://www.web-strategist.com/blog/2012/05/19/facebook-a-brilliant-business-model/}{http://www.web-strategist.com/blog/2012/05/19/facebook-a-brilliant-business-model/}

\bibitem{lyons2012}
Dan Lyons, 2012, “Facebook has a Business model problem”, 14/11 - 2012,
\href{http://readwrite.com/2012/11/14/facebook-has-a-business-model-problem\#awesm=~oAtfMXJb5FLH5V}{http://readwrite.com/2012/11/14/facebook-has-a-business-model-problem\#awesm=~oAtfMXJb5FLH5V}

\bibitem{dekel2013}
Elan Dekel, 2013, “Facebook pages are a bad investment for small
businesses”,22/1 – 2013,
\href{http://www.forbes.com/sites/elandekel/2013/01/22/facebook-pages-are-a-bad-investment-for-small-businesses/}{http://www.forbes.com/sites/elandekel/2013/01/22/facebook-pages-are-a-bad-investment-for-small-businesses/}

\bibitem{facebookbusiness}
Facebook, https://www.facebook.com/business/overview

\bibitem{ariaprimer}
    \href{http://www.w3.org/TR/wai-aria-primer/\#problems}{http://www.w3.org/TR/wai-aria-primer/\#problems }

\bibitem{buzzi2010}
Buzzi, M.C.; Buzzi, M.; Leporini, B.; Akhter, F., "Is Facebook really "open" to
all?," Technology and Society (ISTAS), 2010 IEEE International Symposium on ,
vol., no., pp.327,336, 7-9 June 2010
URL: \href{http://ieeexplore.ieee.org/stamp/stamp.jsp?tp=\&arnumber=5514621\&isnumber=5514600}{http://ieeexplore.ieee.org/stamp/stamp.jsp?tp=\&arnumber=5514621\&isnumber=5514600}

\bibitem{ariaspec}
    \href{http://www.w3.org/TR/wai-aria/}{http://www.w3.org/TR/wai-aria/ }

\bibitem{hailpern2009}
Joshua Hailpern, Loretta Guarino-Reid, Richard Boardman, and Srinivas Annam.
2009. Web 2.0: blind to an accessible new world.  In <em>Proceedings of the
18th international conference on World wide web</em> (WWW '09). ACM, New York,
NY, USA,  821-830. DOI=10.1145/1526709.1526820
\href{http://doi.acm.org/10.1145/1526709.1526820}{http://doi.acm.org/10.1145/1526709.1526820}

\bibitem{petrie2004}
Helen Petrie, Fraser Hamilton, and Neil King. 2004. Tension, what tension?:
Website accessibility and visual design.  In <em>Proceedings of the 2004
international cross-disciplinary workshop on Web accessibility (W4A)</em> (W4A
'04). ACM, New York, NY, USA,  13-18. DOI=10.1145/990657.990660
\href{http://doi.acm.org/10.1145/990657.990660}{http://doi.acm.org/10.1145/990657.990660}

\bibitem{meiselwitz2009}
Gabriele Meiselwitz and Jonathan Lazar. 2009. Accessibility of Registration
Mechanisms in Social Networking Sites.  In <em>Proceedings of the 3d
International Conference on Online Communities and Social Computing: Held as
Part of HCI International 2009</em> (OCSC '09), A. Ant Ozok and Panayiotis
Zaphiris (Eds.). Springer-Verlag, Berlin, Heidelberg,  82-90.
DOI=10.1007/978-3-642-02774-1\_9 \href{http://dx.doi.org/10.1007/978-3-642-02774-1\_9}{http://dx.doi.org/10.1007/978-3-642-02774-1\_9}

\bibitem{jaws}
Freedom Scientific. JAWS for Windows ® Screen Reading Software.
\href{http://www.freedomscientific.com/products/fs/jaws-product-page.asp}{http://www.freedomscientific.com/products/fs/jaws-product-page.asp }
4/4 2014)

\end{thebibliography}

\end{document}


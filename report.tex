
%% report_template.tex
%% V1.0
%% 2012-03-16
%% by Jesper Pedersen Notander
%% See:
%% http://www.cs.lth.se/jesper_pedersen_notander
%% for current contact information.
%% V1.1
%% 2014-02-20
%% by Per Runeson
%% See:
%% http://www.cs.lth.se/per_runeson
%% for current contact information.
%%
%% This is a template file contaning instructions and a skeleton outline 
%% for the final report in the course ETSA05: Software Engineering 
%% Process - Soft Issues, given by the Department of Computer Science at 
%% Lund University, Sweden.
%% 
%% This template requires IEEEtran.cls, written by Michael Shell, version 
%% 1.7 or later.
%%
%% Support sites:
%% http://www.cs.lth.se/etsa05/
%% http://www.ieee.org/

%%*************************************************************************
%% Legal Notice:
%% This code is offered as-is without any warranty either expressed or
%% implied; without even the implied warranty of MERCHANTABILITY or
%% FITNESS FOR A PARTICULAR PURPOSE! 
%%
%% User assumes all risk.
%%
%% In no event shall Lund University or any contributor to this code be 
%% liable for any damages or losses, including, but not limited to, 
%% incidental, consequential, or any other damages, resulting from the use 
%% or misuse of any information contained here.
%%
%% All comments are the opinions of their respective authors and are not
%% necessarily endorsed by Lund University.
%%
%% This work is distributed under the LaTeX Project Public License (LPPL)
%% ( http://www.latex-project.org/ ) version 1.3, and may be freely used,
%% distributed and modified. A copy of the LPPL, version 1.3, is included
%% in the base LaTeX documentation of all distributions of LaTeX released
%% 2003/12/01 or later.
%%
%% Retain all contribution notices and credits.
%% ** Modified files should be clearly indicated as such, including  **
%% ** renaming them and changing author support contact information. **
%%
%% File list of work: report_template.tex
%%*************************************************************************


\documentclass[conference]{IEEEtran}
% If IEEEtran.cls has not been installed into the LaTeX system files,
% manually specify the path to it like:
% \documentclass[conference]{../sty/IEEEtran}

\begin{document}

\title{Outline}


% author names and affiliations
% use a multiple column layout for up to three different
% affiliations
\author{\IEEEauthorblockN{Name/s per 1st Affiliation (Author)}
\IEEEauthorblockA{line 1 (of Affiliation): dept. name of organization\\
line 2: name of organization, acronyms acceptable\\
line 3: City, Country\\
line 4: e-mailaddress if desired}
\and
\IEEEauthorblockN{Per Olson\\and Nina Garcic\\}
\IEEEauthorblockA{Student at Lund University\\
Lund, Sweden\\
me@e-mail.me}}


\maketitle


\begin{abstract}
Facebook has over the last decade segmented itself as the leading social
network on the internet.
A service with the goal of bringing people all over the world together in an
effortless way. This paper aims to summarize the social aspects of Facebook
from a few of the aspects: user-friendliness, financial, ethical and legal. 
Method for the study, results and conclusions must be written. 
\end{abstract}

\section{Introduction}
The introduction section can be used to introduce the company in general or to
introduce the purpose and context of this report. This template is a document
that provides the predefined outline of your group essay from the seminar
compendium. If you want to change or adjust your outline, you must do so before
the outline is due (April 11, 2014). The outline shall include a brief
description of your system in Section~\ref{system}, and the results from the
seminars in Sections~\ref{quality}, \ref{availability}, and \ref{legal} --- at
least an itemized form. Further, literature that you base your analysis on
should be listed. Primarily, \emph{peer-reviewed literature} shall be used.

It is important to note that you have a page limit of 5-7 pages (excluding references) for your final report (which is due May 16, 2014), hence it is important to decide how much should be analyzed, discussed, and written for each section.

\section{Description of the System}
\label{system}

Facebook is a social networking platform created by Mark Zuckerberg in 2004.
Facebook originally only accepted students attending Harvard but has today
expanded, allowing user from all over the world. Today facebook has
approximately 1.2 billion monthly users reference according to compete.com and
has now taken the spot as the number one sociall network of the world.

Facebook is a way for people to share thoughts, pictures and video to friends
and family by posting them on your personal profile. Your profile also contains
a so called “News feed” where updates posted by people on your friends list
will be displayed. Users are also able to create fan pages, these are open
pages and liking one of these pages makes updates from said page show up on
your news feed without. This gives people with the same interests a way to
share ideas and opinions about a subject.


Facebook also features several other well used features like:
“The Facebook chat”: where about  3 million messages are sent every minute.
An event calendar: which is a digital calendar where you are able to create and
invite people to events. You can also create public events that are open for
everyone to join.

//Kvar att skriva om: Appar, grupper(?) lite mer almänt om vad facebook kan
användas till. Exempelvis för att organisera händelser osv.  


\section{Quality Characteristics}
\label{quality}
From seminar 1 we considered the main characteristics for facebook to be
usability, functionality and portability and as we’re doing more research this
seems to hold true. We will give further understanding as to why this is the
case, by focusing on what users prioritize when using facebook. At this point
facebook has grown so large, that people will use the product regardless if it
has a few bugs. We will also give examples on how facebook is faulty in other
characteristics, mainly the reliability, security and understandability(Thomas
Paul et.al).


\section{Availability for Disabled}
\label{availability}
we’ve decided to focus mainly on people with impaired vision. This decision
stems from the result of a user panel study conducted by Petrie et al. the
blind panel group had the lowest success rate of all the disability groups
represented in the study.  

Blind computer users use a piece of software called a screen reader. JAWS is
one example for Windows that seems to be relatively well established. The
purpose of the screen reader is to read the content on screen. The main
challenge in this is to filter the relevant information from the irrelevant,
something done automatically by a seeing user at a quick glance on the screen.
The blind user is however completely dependent on the feedback provided by
their screen reader for interacting with other software on the computer. Most
desktop applications of relevance such as office packages like Microsoft Word
come with support for screen reader. however with the advance of web 2.0
technology more and more applications that used to be desktop-based are now
provided by online services. The functionality provided by these websites is to
a large extent powered by JavaScript which provides dynamic updates of the web
page. it increases the performance of web pages by not making the browser
reload the whole page when something changes. 

The problem\cite{jung2004} is that dynamic sites are often dependent on
elements that contain no semantic information, such as DIV-tags, these are then
decorated by javascript to provide functionality. The main technology for
adapting screen readers to dynamically generated content is the W3C’s WAI-ARIA
standards. Without semantic information, how are the screen readers supposed to
interpret a given DIV tag? This is where WAI-ARIA comes in. It defines an
interface for providing semańtic information through the addition of more
classes to the DOM.  this information is then parsed by the client and
presented to the screen reader in a meaningful way.

\section{Ethical Aspects}
\label{ethics}
As previously mentioned Facebook has approximately  1.2 billion monthly users
which means Facebook has access to names, email-address, date of birth and your
gender since this is mandatory information for signing up a Facebook account.
Most people also submit their phone numbers and addresses. So what is Facebook
actually allowed to do with this information?

What can you post on Facebook?
In Italy, pro maffia groups were discovered which caused the Italian senate to
pass a bill that says that Italian ISP’s have to block access to websites
refusing to remove content that might be used to plan criminal activities. Both
Facebook and Google have criticized this decision saying this would be against
the freedoms of speech if the person in question actually wasn't planning
anything illegal.  Having this law passed would as Facebook and Google pointed
out enable the government to censor whoever they would want by claiming that
the person was planning something illegal. 

More to discuss:
What can fb do with your information, who can see what? privacy policy, crime,
online violence.

\section{Legal Aspects}
\label{legal}
In this section we plan on writing about the intellectual properties of
facebook. Being the leading company in its field, facebook has a large focus on
searching the web for violations of their intellectual property. This concern
originates from the fact that software code is a very difficult thing to
protect. According to  Xuan-Thao Nguyen et. al. this leads to facebook
protecting its software code through trade dress lawsuits and so on. We will
continue by providing examples of the different lawsuits that facebook has gone
through over the last couple of years, and lastly an overview of how facebook
protects itself from the information that their users post will be given.

\section{Business Aspects}
\label{financial}
Facebook is a global network which is build up in a way to create utility for
users, advertisers and developers. It focuses on providing a social network
where people can stay connected with friends and family and is a place to share
what they want to express, what they have been up to or just to be kept posted
of others profiles. 

The breakthrough facebook has successfully achieved is something Mark
Zuckerberg could not have suspected when implemented the minimum social network
as the facebook was at first. The idea to build a network where people could
express themselves in different ways and connect to each other was the
originally idea that rapidly expanded to something much bigger. Now the
business that facebook is built on does not look the same as it once did.
Facebook has now provided a platform for users and developers to build
applications and Websites that integrate with the huge global network that
facebook now is. Ads put up by advertisers can be seen by 1,23 billion monthly
active users, according to facebook newsroom which is giving facebook a revenue
that supposedly will cover their expenses.

\section{Summary}
In the summary section you should summarize what your report is about and
present your main findings/consequences. 

\section{Contribution Statement}
In this section you state the role and contribution of each co-author. 

% references section
\begin{thebibliography}{1}

\bibitem{jung2004}
H-W. Jung, S-G Kim, and C-S Chung, ''Measuring Software Product Quality: A
Survey of ISO/IEC 9126'' IEEE Software, 21(5), pp. 88--92, 2004.

\bibitem{berenbach2004}
B. Berenbach, and M. Broy, ''Professional and ethical dilemmas in software
engineering'' Computer, 42(1), pp. 74--88, 2004

\bibitem{nguyen2012}
Xuan-Thao Nguyen, and Jeffrey A. Maine, “Taxing facebook Code: Debugging the
code and software”, Buffalo Law Review, January 2012(Peer Reviewed)

\bibitem{lilley2012}
Stephen Lilley, Frances S. Grodzinsky, Andra Gumbus, (2012) "Revealing the
commercialized and compliant Facebook user", Journal of Information,
Communication and Ethics in Society, Vol. 10 Iss: 2, pp.82 – 92 (Peer Reviewed)

\bibitem{cabrera2011}
Jan André Blackburn-Cabrera, “The war on Facebook: Privacy on social networks”,
International Journal of Liability and Scientific Enquiry(IJLSE), Vol. 4, No.
4, 2011 (Peer Reviewed)

\bibitem{hensley2012}
Jason Hensley, 2012, “Facebook success shows why Intellectual Property laws are
stifling”,6/2 - 2012,
http://www.policymic.com/articles/3939/facebook-success-shows-why-intellectual-property-laws-are-stifling

\bibitem{brodkin2012}
Jon Brodkin,  2012, “Yahoo IP lawsuit: We patented “Facebook’s entire social
network model”, 13/3 - 2012,
http://arstechnica.com/tech-policy/2012/03/yahoo-patent-lawsuit-we-invented-facebooks-entire-social-network-model/

\bibitem{delamerced2012}
Michael J.De La Merced, 2012, “Yahoo and Facebook settle patent lawsuits”, 6/6
– 2012,
http://dealbook.nytimes.com/2012/07/06/yahoo-and-facebook-said-to-settle-patent-lawsuits/?_php=true&_type=blogs&_php=true&_type=blogs&_r=1

\bibitem{epo2013}
European Patent Office, 2013, “Patents for software? European law and
practice”, 26/11 – 2013, http://www.epo.org/news-issues/issues/software.html

\bibitem{facebookrights2013}
Facebook, - , “Statement of Rights and Responsibilities”, 15/11 - 2013
https://www.facebook.com/legal/terms

\bibitem{yulianti2013}
Ida Yulianti, and Wei Tung, 2013, “The Relationship Among Brand Experience,
Brand Image and

\bibitem{indonesia}
Customer Satisfaction of Facebook Users in Indonesia“, 1818-4952, Department of
Management, Faculty of Economic and Business, and University of Brawijaya 1

\bibitem{chiayi}
Graduate Institute of Marketing and Logistics, National Chiayi University,
http://www.idosi.org/wasj/wasj28(efmo)13/15.pdf (Peer Reviewed)

\bibitem{paul}
Thomas Paul, Daniel Puscher, and Thorsten Strufe, “Improving the Usability of
Privacy settings in Facebook”, http://arxiv.org/abs/1109.6046 (Peer Reviewed)

\bibitem{kumparak2013}
Greg Kumparak, 2013, “Security Researcher Hacks Mark Zuckerberg’s Wall To Prove
His Exploit Works”, 18/8 - 2013,
http://techcrunch.com/2013/08/18/security-researcher-hacks-mark-zuckerbergs-wall-to-prove-his-exploit-works/

\bibitem{priestley2011}
*Is this a valid source?*Evan Priestley, 2011, “Is it true that facebook has no
testers”, 20/12 – 2011,
http://www.quora.com/Is-it-true-that-Facebook-has-no-testers

\bibitem{priestley2011b}
*Is this a valid source?*Evan Priestley, 2011, “Why does Facebook not need to
produce particulary high-quality code?”, 20/12 – 2011,
http://www.quora.com/Facebook-Engineering/Why-does-Facebook-not-need-to-produce-particularly-high-quality-software

\bibitem{innovation}
(?)Buisness Model Innovation Matters,
http://bmimatters.com/2012/01/08/understanding-business-model-fundamentals/

\bibitem{weiner2010}
Nico Weiner, 2010 , “Social networks evolving into service platforms - the

\bibitem{leipzig2010}
Facebook-case from a business model viewpoint”, 40th Jahrestagung der
Gesellschaft fur Informatik e.V. (GI): Service Science - Neue Perspektiven fur
die Informatik, INFORMATIK 2010, Leipzig, Germany, September 27, 2010 - October
1, 2010, http://subs.emis.de/LNI/Proceedings/Proceedings175/527.pdf

\bibitem{weiner2011}
Nico Weiner, 2011, “A Business Model Framework for the Design and Evaluation of
Business Models in 

\bibitem{srii}
the Internet of Services ”, Annual SRII Global Conference, Stuttgart, Germany,
-,
http://ieeexplore.ieee.org.ludwig.lub.lu.se/stamp/stamp.jsp?tp=&arnumber=5958069&tag=1

\bibitem{hof2011}
Robert D. Hof, 2011, “You are the ad”, 1099-274X, -,
http://eds.a.ebscohost.com.ludwig.lub.lu.se/eds/detail?sid=61403bdf-588f-4bf7-be69-d67225c27c5b%40sessionmgr4001&vid=3&hid=4208&bdata=JnNpdGU9ZWRzLWxpdmUmc2NvcGU9c2l0ZQ%3d%3d#db=bth&AN=61481061

\bibitem{ortutay2014}
Barbara Ortutay, 2014, “Facebook: Born in a dorm room, the site that had no
business plan turns 10 today and is worth \$150 billion”, 4/2 – 2014
,http://www.vancouversun.com/touch/story.html?id=9468642

\bibitem{haigh2013}
Michel M. Haigh, Pamela Brubaker, Erin Whiteside, (2013) "Facebook: examining
the information presented and its impact on stakeholders", Corporate
Communications: An International Journal, Vol. 18 Iss: 1, pp.52 – 69

\bibitem{krombholz2012}
Katharina Krombholz et. Al, 2012, “Fake identities in social media: A case
study on the sustainability of the Facebook business model”, 2093-0739, SBA,
Research, Vienna, Austria (et. al)

\bibitem{mourdoukoutas}
Panos Mourdoukoutas, 2014, “What's wrong with Facebook's Business model and
Innovation strategy?”, 23/2 - 2014,
http://www.forbes.com/sites/panosmourdoukoutas/2014/02/23/whats-wrong-with-facebooks-business-model-and-innovation-strategy/

\bibitem{dixon2012}
Chris Dixon, 2012, “Facebook's business model”, 15/5 – 2012,
http://www.cdixon.org/2012/05/15/facebooks-business-model/

\bibitem{owyang2012}
Jeremiah Owyang, 2012, “Facebook: A brilliant Business model”, 19/5 - 2012,
http://www.web-strategist.com/blog/2012/05/19/facebook-a-brilliant-business-model/

\bibitem{lyons2012}
Dan Lyons, 2012, “Facebook has a Business model problem”, 14/11 - 2012,
http://readwrite.com/2012/11/14/facebook-has-a-business-model-problem#awesm=~oAtfMXJb5FLH5V

\bibitem{dekel2013}
Elan Dekel, 2013, “Facebook pages are a bad investment for small
businesses”,22/1 – 2013,
http://www.forbes.com/sites/elandekel/2013/01/22/facebook-pages-are-a-bad-investment-for-small-businesses/

\bibitem{facebookbusiness}
Facebook, https://www.facebook.com/business/overview

\bibitem{ariaprimer}
http://www.w3.org/TR/wai-aria-primer/#problems (accessed 4/4 2014)

\bibitem{buzzi2010}
Buzzi, M.C.; Buzzi, M.; Leporini, B.; Akhter, F., "Is Facebook really "open" to
all?," Technology and Society (ISTAS), 2010 IEEE International Symposium on ,
vol., no., pp.327,336, 7-9 June 2010
URL: http://ieeexplore.ieee.org/stamp/stamp.jsp?tp=&arnumber=5514621&isnumber=5514600

\bibitem{ariaspec}
http://www.w3.org/TR/wai-aria/ (accessed 4/4 2014)

\bibitem{hailpern2009}
Joshua Hailpern, Loretta Guarino-Reid, Richard Boardman, and Srinivas Annam.
2009. Web 2.0: blind to an accessible new world.  In <em>Proceedings of the
18th international conference on World wide web</em> (WWW '09). ACM, New York,
NY, USA,  821-830. DOI=10.1145/1526709.1526820
http://doi.acm.org/10.1145/1526709.1526820

\bibitem{petrie2004}
Helen Petrie, Fraser Hamilton, and Neil King. 2004. Tension, what tension?:
Website accessibility and visual design.  In <em>Proceedings of the 2004
international cross-disciplinary workshop on Web accessibility (W4A)</em> (W4A
'04). ACM, New York, NY, USA,  13-18. DOI=10.1145/990657.990660
http://doi.acm.org/10.1145/990657.990660

\bibitem{meiselwitz2009}
Gabriele Meiselwitz and Jonathan Lazar. 2009. Accessibility of Registration
Mechanisms in Social Networking Sites.  In <em>Proceedings of the 3d
International Conference on Online Communities and Social Computing: Held as
Part of HCI International 2009</em> (OCSC '09), A. Ant Ozok and Panayiotis
Zaphiris (Eds.). Springer-Verlag, Berlin, Heidelberg,  82-90.
DOI=10.1007/978-3-642-02774-1_9 http://dx.doi.org/10.1007/978-3-642-02774-1_9

\bibitem{jaws}
Freedom Scientific. JAWS for Windows ® Screen Reading Software.
http://www.freedomscientific.com/products/fs/jaws-product-page.asp (accessed
4/4 2014)

\end{thebibliography}

\end{document}



%% report_template.tex
%% V1.0
%% 2012-03-16
%% by Jesper Pedersen Notander
%% See:
%% http://www.cs.lth.se/jesper_pedersen_notander
%% for current contact information.
%% V1.1
%% 2014-02-20
%% by Per Runeson
%% See:
%% http://www.cs.lth.se/per_runeson
%% for current contact information.
%%
%% This is a template file contaning instructions and a skeleton outline 
%% for the final report in the course ETSA05: Software Engineering 
%% Process - Soft Issues, given by the Department of Computer Science at 
%% Lund University, Sweden.
%% 
%% This template requires IEEEtran.cls, written by Michael Shell, version 
%% 1.7 or later.
%%
%% Support sites:
%% http://www.cs.lth.se/etsa05/
%% http://www.ieee.org/

%%*************************************************************************
%% Legal Notice:
%% This code is offered as-is without any warranty either expressed or
%% implied; without even the implied warranty of MERCHANTABILITY or
%% FITNESS FOR A PARTICULAR PURPOSE! 
%%
%% User assumes all risk.
%%
%% In no event shall Lund University or any contributor to this code be 
%% liable for any damages or losses, including, but not limited to, 
%% incidental, consequential, or any other damages, resulting from the use 
%% or misuse of any information contained here.
%%
%% All comments are the opinions of their respective authors and are not
%% necessarily endorsed by Lund University.
%%
%% This work is distributed under the LaTeX Project Public License (LPPL)
%% ( http://www.latex-project.org/ ) version 1.3, and may be freely used,
%% distributed and modified. A copy of the LPPL, version 1.3, is included
%% in the base LaTeX documentation of all distributions of LaTeX released
%% 2003/12/01 or later.
%%
%% Retain all contribution notices and credits.
%% ** Modified files should be clearly indicated as such, including  **
%% ** renaming them and changing author support contact information. **
%%
%% File list of work: report_template.tex
%%*************************************************************************


\documentclass[conference]{IEEEtran}
% If IEEEtran.cls has not been installed into the LaTeX system files,
% manually specify the path to it like:
% \documentclass[conference]{../sty/IEEEtran}
\usepackage{hyperref}
\begin{document}

\title{Outline}


% author names and affiliations
% use a multiple column layout for up to three different
% affiliations
\author{\IEEEauthorblockN{Kasper Bratz}
\IEEEauthorblockA{Department of Computer Science\\
Faculty of engineering, LTH
Lund, Sweden\\
dat12kbr@student.lu.se}\and

\IEEEauthorblockN{Axel Martensson}
\IEEEauthorblockA{Department of Computer Science\\
Faculty of engineering, LTH
Lund, Sweden\\
dat12ama@student.lu.se}\and

\IEEEauthorblockN{Christine Boghammar}
\IEEEauthorblockA{Department of Computer Science\\
Faculty of engineering, LTH
Lund, Sweden\\
dat12cbo@student.lu.se}\and

\IEEEauthorblockN{Johan Nilsson}
\IEEEauthorblockA{Department of Computer Science\\
Faculty of engineering, LTH
Lund, Sweden\\
dat12jn1@student.lu.se}}


\maketitle


\begin{abstract}
Facebook has over the last decade segmented itself as the leading social
network on the internet. This paper discusses the accessibility, ethical, legal
and business aspects of facebook. 

Method for the study, results and conclusions must be written. 

\end{abstract}

\section{Introduction}
-
\section{Description of the System}
\label{system}

Facebook is a social networking platform created by Mark Zuckerberg in 2004.
Facebook originally only accepted students attending Harvard but has today
expanded, allowing user from all over the world. Today facebook has
approximately 1.2 billion monthly users reference according to compete.com and
has now taken the spot as the number one sociall network of the world.

Facebook is a way for people to share thoughts, pictures and video to friends
and family by posting them on your personal profile. Your profile also contains
a so called “News feed” where updates posted by people on your friends list
will be displayed. Users are also able to create fan pages, these are open
pages and liking one of these pages makes updates from said page show up on
your news feed without. This gives people with the same interests a way to
share ideas and opinions about a subject.


Facebook also features several other well used features like:
“The Facebook chat”: where about  3 million messages are sent every minute.
An event calendar: which is a digital calendar where you are able to create and
invite people to events. You can also create public events that are open for
everyone to join.

//Kvar att skriva om: Appar, grupper(?) lite mer almänt om vad facebook kan
användas till. Exempelvis för att organisera händelser osv.  


\section{Quality Characteristics}
\label{quality}
From seminar 1 we considered the main characteristics for facebook to be
usability, functionality and portability and as we’re doing more research this
seems to hold true. We will give further understanding as to why this is the
case, by focusing on what users prioritize when using facebook. At this point
facebook has grown so large, that people will use the product regardless if it
has a few bugs. We will also give examples on how facebook is faulty in other
characteristics, mainly the reliability, security and understandability(Thomas
Paul et.al).


\section{Availability for Disabled}
\label{availability}
%:set fo=tqa

% Why are we focusing on the blind?
% What types of accessibility products exist for the blind?
% Which problems are encountered by people accessing facebook with a screen reader today?
% What can be done to improve accessibility of facebook when using a screen reader?

This section will focus on the availability of facebook for blind computer
users. The decision to do this stems from the result of a study conducted by
Petrie et al.\cite{petrie2004}. The study was performed using a user panel that
evaluated the accessibility of 100 web pages.  The user panel was comprised of
people with various disabilities affecting their access to the web, such as
blindness and dyslexia. User panel members were told to evaluate 10 websites
and perform two representative tasks on each of them. The blind panel group had
the lowest success rate of all the disability groups represented in the study.
Their success rate was 53\% compared to an average success rate of 82\%. 

Blind computer users use a piece of software called a screen reader.
JAWS\cite{jaws} is one example for Windows users. It has a navigation mode for
browsing the web and an editing mode for filling out forms. The purpose of the
screen reader is to read the text currently being displayed on the screen and
then convey this information via speech synthesis to the computer user. Screen
readers read a page of text by simply starting to read the page sequentially
from left to right, top to bottom. The user then has to use this information to
build a mental model of what is presently on the screen and how it is laid out.
The blind user is of course completely dependent on the feedback provided by
their screen reader for interacting with software on the computer. 

Screen readers have not been able to keep up with the advance of web 2.0
technology and sites like Facebook that makes heavy use of it. The
functionality provided by these websites is to a large extent powered by
JavaScript. Common obstacles for screen readers that arise from the use of
Javacript are custom controls\cite{ariaprimer} that are built with DOM-elements
that contain no semantic information, such as DIV-tags. These are then
decorated by javascript to provide functionality.  However, without semantic
information, how is the screen reader supposed to interpret a given DIV tag?  

The W3C’s WAI-ARIA standard\cite{ariaspec} could be used to solve this
problem\cite{ariaprimer}.  It defines an interface for providing semantic
information through the addition of more classes to the DOM.  The standard is
very new and it is referenced at least once in all the papers cited in this
section of this paper. None of the papers provide any data on its use.
Furthermore, the standard  is not implemented in JAWS. Its most relevant
contribution for the sake of this discussion is that it introduces the notions
of role and state for elements of the DOM.  Consider the case where an element
looks like a link and/or behave like a link, it should then have the WAI-ARIA
role `link' to enable screen readers to interpret it as such.  Hailpern et
al\cite{hailpern2009}.  argue that to make custom control elements accessible
to screen readers, they should have a keyboard listener, be added to the
tabindex and have an appropriate WAI-ARIA role.

JavaScript enables dynamic updates of web pages. This means that it increases
the performance of web applications by not making the browser reload the whole
web page when something changes but instead limiting the reload to a certain
region of the page. Screen readers have not yet caught up with this shift from
static web pages with only text links and a few images here and there to
dynamic web content. They do not detect changes that happen dynamically to the
content (i.e after a page has loaded)\cite{hailpern2009}. When an event occurs
on Facebook, e.g a new friend request is recieved, the user is notified by a
little red icon that appears on the friend button in the topmost navigation
bar. This change is not detected by the screen reader\cite{buzzi2010} because
it is made dynamically by a script.  The only time that an existing screen
reader detects a change is when doing a page refresh. A side effect of doing a
refresh is that the screen reader loses its current place on the page and the
user has to re-read the whole page to get to where the change occurred. The
consequence for the user is that his/her mental model of the web page gets
disturbed and has to be constantly re-evaluated due to the lack of feedback.
This can be a great source of frustration for the users according to
Petrie\cite{petrie2004}.

Support for dynamic updates can be achieved by a combination of two
things\cite{hailpern2009}: First, developers must provide audio feedback on
event, e.g a voice saying ''You've got a new friend request'' when someone
sends you a friend request. Secondly, WAI-ARIA has a concept called ``Live
region''\cite{ariaspec} that enables web developers to mark certain elements
(e.g a table) as dynamically updating. This is meant to enable the screen
reader to keep an eye out for changes happening dynamically in this region of
the page.

To provide a convenient way to access Facebook and other web 2.0 sites,
Hailpern et al\cite{hailpern2009}.  propose an alternative screen reader use
model with a reading mode and a control mode instead of the modes mentioned
above. upon receiving a new private message on Facebook use control mode to
navigate to the inbox and open the message, then use reading mode to read the
message.  We argue that the main advantage of this design is the increased
cohesion between common application design patterns such as
Model-View-Controller and the use model of the screen reader, thereby making
the screen reader more intuitive to use on a web 2.0 site. 

Facebook is a large web site and there are different parts of it that are more
or less accessible. A more nuanced impression of facebook's approach to
accessibility is given in Meiselwitz' and Lazar's \cite{meiselwitz2009} study
of the accessibility of Facebook's sign-up process. They conclude that the
sign-up process is ``moderately inaccessible'' and that it only violates a
single paragraph in Section 508 of the US Rehabilitation Act. While signing up
may be easy, we nevertheless conclude from our review of the literature that
Facebook still has a long way to go before it can be comfortably deemed
accessible for blind computer users.


% % Problems Dynamic content updates: Notifying the user of a change when
% dynamically updating a web page. e.g a new friend request. Now: screen
% readers can only detect that something has changed on page refresh.
% Furthermore when user is not provided feedback by screen reader when
% performing an action such as clicking a link.  This results in the user
% having to painstakingly re-read the whole page to discover the change.
% 
% Custom controls: decorated links, using divs for various stuff, javascript.
% 
% Managing focus: what is important on this page?
% 
% % Solutions
% 
% Dynamic content updates: Live Regions. Developers provide audio feedback on
% event, e.g a voice saying ''You've got a new friend request'' when someone
% sends you a friend request.  
% 
% Custom controls: should have keyboard listener and be added to the tabindex. If
% it looks like a link and behaves like a link, it should have the ARIA role
% `link'.\cite{hailpern2009}
% 
% Alternative User workflow model.  use a reading mode for traditional web
% browsing and a control mode for interacting with the application instead of
% navigation mode and forms mode\cite{hailpern2009}. e.g upon receiving a new
% private message on Facebook use control mode to locate and open the message,
% then use reading mode to read the message.

\section{Ethical Aspects}
\label{ethics}
As previously mentioned Facebook has approximately  1.2 billion monthly users
which means Facebook has access to names, email-address, date of birth and your
gender since this is mandatory information for signing up a Facebook account.
Most people also submit their phone numbers and addresses. So what is Facebook
actually allowed to do with this information?

What can you post on Facebook?
In Italy, pro maffia groups were discovered which caused the Italian senate to
pass a bill that says that Italian ISP’s have to block access to websites
refusing to remove content that might be used to plan criminal activities. Both
Facebook and Google have criticized this decision saying this would be against
the freedoms of speech if the person in question actually wasn't planning
anything illegal.  Having this law passed would as Facebook and Google pointed
out enable the government to censor whoever they would want by claiming that
the person was planning something illegal. 

More to discuss:
What can fb do with your information, who can see what? privacy policy, crime,
online violence.

\section{Legal Aspects}
\label{legal}
In this section we plan on writing about the intellectual properties of
facebook. Being the leading company in its field, facebook has a large focus on
searching the web for violations of their intellectual property. This concern
originates from the fact that software code is a very difficult thing to
protect. According to  Xuan-Thao Nguyen et.\ al.\ this leads to facebook
protecting its software code through trade dress lawsuits and so on. We will
continue by providing examples of the different lawsuits that facebook has gone
through over the last couple of years, and lastly an overview of how facebook
protects itself from the information that their users post will be given.

\section{Business Aspects}
\label{financial}
Facebook is a global network which is build up in a way to create utility for
users, advertisers and developers. It focuses on providing a social network
where people can stay connected with friends and family and is a place to share
what they want to express, what they have been up to or just to be kept posted
of others profiles. 

The breakthrough facebook has successfully achieved is something Mark
Zuckerberg could not have suspected when implemented the minimum social network
as the facebook was at first. The idea to build a network where people could
express themselves in different ways and connect to each other was the
originally idea that rapidly expanded to something much bigger. Now the
business that facebook is built on does not look the same as it once did.
Facebook has now provided a platform for users and developers to build
applications and Websites that integrate with the huge global network that
facebook now is. Ads put up by advertisers can be seen by 1,23 billion monthly
active users, according to facebook newsroom which is giving facebook a revenue
that supposedly will cover their expenses.

\section{Summary}
-
\section{Contribution Statement}
Johan Nilsson - Quality Characteristics, Legal aspects.

Christine Boghammar – Business aspects.

Axel Mårtensson - Availability for Disabled

Kasper Bratz - Ethical aspects, Description of the System

Abstract
Introduction
Summary
% references section
\begin{thebibliography}{1}

\bibitem{jung2004}
H-W. Jung, S-G Kim, and C-S Chung, ''Measuring Software Product Quality: A
Survey of ISO/IEC 9126'' IEEE Software, 21(5), pp. 88--92, 2004.

\bibitem{berenbach2004}
B. Berenbach, and M. Broy, ''Professional and ethical dilemmas in software
engineering'' Computer, 42(1), pp. 74--88, 2004

\bibitem{nguyen2012}
Xuan-Thao Nguyen, and Jeffrey A. Maine, “Taxing facebook Code: Debugging the
code and software”, Buffalo Law Review, January 2012(Peer Reviewed)

\bibitem{lilley2012}
Stephen Lilley, Frances S. Grodzinsky, Andra Gumbus, (2012) "Revealing the
commercialized and compliant Facebook user", Journal of Information,
Communication and Ethics in Society, Vol. 10 Iss: 2, pp.82 – 92 (Peer Reviewed)

\bibitem{cabrera2011}
Jan André Blackburn-Cabrera, “The war on Facebook: Privacy on social networks”,
International Journal of Liability and Scientific Enquiry(IJLSE), Vol. 4, No.
4, 2011 (Peer Reviewed)

\bibitem{hensley2012}
Jason Hensley, 2012, “Facebook success shows why Intellectual Property laws are
stifling”,6/2 - 2012,
\href{http://www.policymic.com/articles/3939/facebook-success-shows-why-intellectual-property-laws-are-stifling}{http://www.policymic.com/articles/3939/facebook-success-shows-why-intellectual-property-laws-are-stifling}

\bibitem{brodkin2012}
Jon Brodkin,  2012, “Yahoo IP lawsuit: We patented “Facebook’s entire social
network model”, 13/3 - 2012,
\href{http://arstechnica.com/tech-policy/2012/03/yahoo-patent-lawsuit-we-invented-facebooks-entire-social-network-model/}{http://arstechnica.com/tech-policy/2012/03/yahoo-patent-lawsuit-we-invented-facebooks-entire-social-network-model/}

\bibitem{delamerced2012}
Michael J.De La Merced, 2012, “Yahoo and Facebook settle patent lawsuits”, 6/6
– 2012,
\href{http://dealbook.nytimes.com/2012/07/06/yahoo-and-facebook-said-to-settle-patent-lawsuits/?\_php=true\&\_type=blogs\&\_php=true\&\_type=blogs\&\_r=1}{http://dealbook.nytimes.com/2012/07/06/yahoo-and-facebook-said-to-settle-patent-lawsuits/?\_php=true\&\_type=blogs\&\_php=true\&\_type=blogs\&\_r=1}

\bibitem{epo2013}
European Patent Office, 2013, “Patents for software? European law and
practice”, 26/11 – 2013, \href{http://www.epo.org/news-issues/issues/software.html}{http://www.epo.org/news-issues/issues/software.html}

\bibitem{facebookrights2013}
Facebook, - , “Statement of Rights and Responsibilities”, 15/11 - 2013
https://www.facebook.com/legal/terms

\bibitem{yulianti2013}
Ida Yulianti, and Wei Tung, 2013, “The Relationship Among Brand Experience,
Brand Image and

\bibitem{indonesia}
Customer Satisfaction of Facebook Users in Indonesia“, 1818-4952, Department of
Management, Faculty of Economic and Business, and University of Brawijaya 1

\bibitem{chiayi}
Graduate Institute of Marketing and Logistics, National Chiayi University,
\href{http://www.idosi.org/wasj/wasj28(efmo)13/15.pdf}{http://www.idosi.org/wasj/wasj28(efmo)13/15.pdf }

\bibitem{paul}
Thomas Paul, Daniel Puscher, and Thorsten Strufe, “Improving the Usability of
Privacy settings in Facebook”, \href{http://arxiv.org/abs/1109.6046}{http://arxiv.org/abs/1109.6046 }

\bibitem{kumparak2013}
Greg Kumparak, 2013, “Security Researcher Hacks Mark Zuckerberg’s Wall To Prove
His Exploit Works”, 18/8 - 2013,
\href{http://techcrunch.com/2013/08/18/security-researcher-hacks-mark-zuckerbergs-wall-to-prove-his-exploit-works/}{http://techcrunch.com/2013/08/18/security-researcher-hacks-mark-zuckerbergs-wall-to-prove-his-exploit-works/}

\bibitem{priestley2011}
*Is this a valid source?*Evan Priestley, 2011, “Is it true that facebook has no
testers”, 20/12 – 2011,
\href{http://www.quora.com/Is-it-true-that-Facebook-has-no-testers}{http://www.quora.com/Is-it-true-that-Facebook-has-no-testers}

\bibitem{priestley2011b}
*Is this a valid source?*Evan Priestley, 2011, “Why does Facebook not need to
produce particulary high-quality code?”, 20/12 – 2011,
\href{http://www.quora.com/Facebook-Engineering/Why-does-Facebook-not-need-to-produce-particularly-high-quality-software}{http://www.quora.com/Facebook-Engineering/Why-does-Facebook-not-need-to-produce-particularly-high-quality-software}

\bibitem{innovation}
(?)Buisness Model Innovation Matters,
\href{http://bmimatters.com/2012/01/08/understanding-business-model-fundamentals/}{http://bmimatters.com/2012/01/08/understanding-business-model-fundamentals/}

\bibitem{weiner2010}
Nico Weiner, 2010 , “Social networks evolving into service platforms - the

\bibitem{leipzig2010}
Facebook-case from a business model viewpoint”, 40th Jahrestagung der
Gesellschaft fur Informatik e.V. (GI): Service Science - Neue Perspektiven fur
die Informatik, INFORMATIK 2010, Leipzig, Germany, September 27, 2010 - October
1, 2010, \href{http://subs.emis.de/LNI/Proceedings/Proceedings175/527.pdf}{http://subs.emis.de/LNI/Proceedings/Proceedings175/527.pdf}

\bibitem{weiner2011}
Nico Weiner, 2011, “A Business Model Framework for the Design and Evaluation of
Business Models in 

\bibitem{srii}
the Internet of Services ”, Annual SRII Global Conference, Stuttgart, Germany,
-,
\href{http://ieeexplore.ieee.org.ludwig.lub.lu.se/stamp/stamp.jsp?tp=\&arnumber=5958069\&tag=1}{http://ieeexplore.ieee.org.ludwig.lub.lu.se/stamp/stamp.jsp?tp=\&arnumber=5958069\&tag=1}

\bibitem{hof2011}
Robert D. Hof, 2011, “You are the ad”, 1099-274X, -,
\href{http://eds.a.ebscohost.com.ludwig.lub.lu.se/eds/detail?sid=61403bdf-588f-4bf7-be69-d67225c27c5b\%40sessionmgr4001\&vid=3\&hid=4208\&bdata=JnNpdGU9ZWRzLWxpdmUmc2NvcGU9c2l0ZQ\%3d\%3d\#db=bth\&AN=61481061}{http://eds.a.ebscohost.com.ludwig.lub.lu.se/eds/detail?sid=61403bdf-588f-4bf7-be69-d67225c27c5b\%40sessionmgr4001\&vid=3\&hid=4208\&bdata=JnNpdGU9ZWRzLWxpdmUmc2NvcGU9c2l0ZQ\%3d\%3d\#db=bth\&AN=61481061}

\bibitem{ortutay2014}
Barbara Ortutay, 2014, “Facebook: Born in a dorm room, the site that had no
business plan turns 10 today and is worth \$150 billion”, 4/2 – 2014
,\href{http://www.vancouversun.com/touch/story.html?id=9468642}{http://www.vancouversun.com/touch/story.html?id=9468642}

\bibitem{haigh2013}
Michel M. Haigh, Pamela Brubaker, Erin Whiteside, (2013) "Facebook: examining
the information presented and its impact on stakeholders", Corporate
Communications: An International Journal, Vol. 18 Iss: 1, pp.52 – 69

\bibitem{krombholz2012}
Katharina Krombholz et. Al, 2012, “Fake identities in social media: A case
study on the sustainability of the Facebook business model”, 2093-0739, SBA,
Research, Vienna, Austria (et.\ al)

\bibitem{mourdoukoutas}
Panos Mourdoukoutas, 2014, “What's wrong with Facebook's Business model and
Innovation strategy?”, 23/2 - 2014,
\href{http://www.forbes.com/sites/panosmourdoukoutas/2014/02/23/whats-wrong-with-facebooks-business-model-and-innovation-strategy/}{http://www.forbes.com/sites/panosmourdoukoutas/2014/02/23/whats-wrong-with-facebooks-business-model-and-innovation-strategy/}

\bibitem{dixon2012}
Chris Dixon, 2012, “Facebook's business model”, 15/5 – 2012,
\href{http://www.cdixon.org/2012/05/15/facebooks-business-model/}{http://www.cdixon.org/2012/05/15/facebooks-business-model/}

\bibitem{owyang2012}
Jeremiah Owyang, 2012, “Facebook: A brilliant Business model”, 19/5 - 2012,
\href{http://www.web-strategist.com/blog/2012/05/19/facebook-a-brilliant-business-model/}{http://www.web-strategist.com/blog/2012/05/19/facebook-a-brilliant-business-model/}

\bibitem{lyons2012}
Dan Lyons, 2012, “Facebook has a Business model problem”, 14/11 - 2012,
\href{http://readwrite.com/2012/11/14/facebook-has-a-business-model-problem\#awesm=~oAtfMXJb5FLH5V}{http://readwrite.com/2012/11/14/facebook-has-a-business-model-problem\#awesm=~oAtfMXJb5FLH5V}

\bibitem{dekel2013}
Elan Dekel, 2013, “Facebook pages are a bad investment for small
businesses”,22/1 – 2013,
\href{http://www.forbes.com/sites/elandekel/2013/01/22/facebook-pages-are-a-bad-investment-for-small-businesses/}{http://www.forbes.com/sites/elandekel/2013/01/22/facebook-pages-are-a-bad-investment-for-small-businesses/}

\bibitem{facebookbusiness}
Facebook, https://www.facebook.com/business/overview

\bibitem{ariaprimer}
    \href{http://www.w3.org/TR/wai-aria-primer/\#problems}{http://www.w3.org/TR/wai-aria-primer/\#problems }

\bibitem{buzzi2010}
Buzzi, M.C.; Buzzi, M.; Leporini, B.; Akhter, F., "Is Facebook really "open" to
all?," Technology and Society (ISTAS), 2010 IEEE International Symposium on ,
vol., no., pp.327,336, 7-9 June 2010
URL: \href{http://ieeexplore.ieee.org/stamp/stamp.jsp?tp=\&arnumber=5514621\&isnumber=5514600}{http://ieeexplore.ieee.org/stamp/stamp.jsp?tp=\&arnumber=5514621\&isnumber=5514600}

\bibitem{ariaspec}
    \href{http://www.w3.org/TR/wai-aria/}{http://www.w3.org/TR/wai-aria/ }

\bibitem{hailpern2009}
Joshua Hailpern, Loretta Guarino-Reid, Richard Boardman, and Srinivas Annam.
2009. Web 2.0: blind to an accessible new world.  In <em>Proceedings of the
18th international conference on World wide web</em> (WWW '09). ACM, New York,
NY, USA,  821-830. DOI=10.1145/1526709.1526820
\href{http://doi.acm.org/10.1145/1526709.1526820}{http://doi.acm.org/10.1145/1526709.1526820}

\bibitem{petrie2004}
Helen Petrie, Fraser Hamilton, and Neil King. 2004. Tension, what tension?:
Website accessibility and visual design.  In <em>Proceedings of the 2004
international cross-disciplinary workshop on Web accessibility (W4A)</em> (W4A
'04). ACM, New York, NY, USA,  13-18. DOI=10.1145/990657.990660
\href{http://doi.acm.org/10.1145/990657.990660}{http://doi.acm.org/10.1145/990657.990660}

\bibitem{meiselwitz2009}
Gabriele Meiselwitz and Jonathan Lazar. 2009. Accessibility of Registration
Mechanisms in Social Networking Sites.  In <em>Proceedings of the 3d
International Conference on Online Communities and Social Computing: Held as
Part of HCI International 2009</em> (OCSC '09), A. Ant Ozok and Panayiotis
Zaphiris (Eds.). Springer-Verlag, Berlin, Heidelberg,  82-90.
DOI=10.1007/978-3-642-02774-1\_9 \href{http://dx.doi.org/10.1007/978-3-642-02774-1\_9}{http://dx.doi.org/10.1007/978-3-642-02774-1\_9}

\bibitem{jaws}
Freedom Scientific. JAWS for Windows ® Screen Reading Software.
\href{http://www.freedomscientific.com/products/fs/jaws-product-page.asp}{http://www.freedomscientific.com/products/fs/jaws-product-page.asp }
4/4 2014)

\end{thebibliography}

\end{document}

